\documentclass[11pt]{beamer}
\usetheme{Malmoe}
\usepackage[utf8]{inputenc}
\usepackage{amsmath}
\usepackage{amsfonts}
\usepackage{amssymb}
\usepackage{graphicx}
%\author{}
%\title{Tutorial #4}
%\setbeamercovered{transparent} 
%\setbeamertemplate{navigation symbols}{} 
%\logo{} 
%\institute{} 
%\date{} 
%\subject{} 
\begin{document}

%\begin{frame}
%\titlepage
%\end{frame}

%\begin{frame}
%\tableofcontents
%\end{frame}

%\begin{frame}{Tutorial # 4}
%\end{frame}
\begin{frame}
{\large Tutorial \# 4 - Calculating Significance by Monte Carlo}

In the homework you should have found the correlation coefficients for some data.  
The Pearson Correlation Coefficient is
\begin{align}
r_{xy} = \frac{ \sum_i (x_i - \bar{x} ) (y_i - \bar{y}) }{ \sqrt{ S^2_x S^2_y} }
\end{align}
where
\begin{align}
\bar{x} = \frac{1}{N}\sum_i x_i ~~~~ S^2_x = \frac{1}{(N-1)} \sum_i ( x_i - \bar{x} )
\end{align}
\end{frame}

\begin{frame}
A non-zero $r_{xy}$ should indicate that there is a correlation between $x$ and $y$.
$r_{xy}$ is an estimator of the populations true correlation coefficient
\begin{align}
\rho_{xy} = \frac{C_{XY}}{\sqrt{\sigma^2_x \sigma_y^2} }
\end{align}

But $r_{xy}$ is a function of random variables so we would not expect it to be exactly 0 
even if there were no correlations.  We could try to calculate the probability distribution of 
$r_{xy}$ analytically, but this might be difficult (more on this in lecture).
\end{frame}
\begin{frame}
{\large Monte Carlo}

Instead lets find the significance of your measured $r_{xy}$ by Monte Carlo given the hypothesis that 
$x$ and $y$ are not correlated.

\vspace{1cm}
1) Create two vectors (X and Y) of random normally distributed numbers with variance 1 and mean zero.  Each vector should be 1000 elements long as was the data in the homework.  Calculate $r_{xy}$ for these.  X and Y 
are uncorrelated since they where generated independently.

\vspace{0.9cm}
2) Repeat step 1 a thousand times to get a distribution of $r_{xy}$.  You should to this in a loop.  There is no reason to save all the X's and Y's.

\vspace{0.9cm}
3) Plot a histogram of your  $r_{xy}$ values.  Does it look Gaussian?
\end{frame}
\begin{frame}
4) What is the fraction of times $|r_{xy}|$ is larger than your measured value for the data in file homework\_01\_2d-datafile.csv ?  Would you expect to get this value if there were no correlation?

\vspace{0.9cm}

5) Do the exercise above over but this time use the measured $S^2_x$, $S^2_y$, $\bar{x}$ and $\bar{y}$ from homework\_01\_2d-datafile.csv  to generate the  X and Y variables.   Plot the new histogram of $r_{xy}$ over the old one.   Is there a difference or is the distribution of $r_{xy}$ independent of the averages and variances of the distributions?
\end{frame}
\begin{frame}
6) Order your sample of $r_{xy}$'s from smallest to largest.  Take the $i$th value to be an estimate of the $r_{xy}$ where $(N-i)/N$ of the probability distribution is larger than it.  For example the 95\% upper bound would be at $i/N = 0.95$.  What is the 95\% upper bound on $r_{xy}$ if there is no correlation between $X$ and $Y$?  We will call this $r_{0.95}$.
In lecture we will find that the variance in this estimator is
\begin{align}
Var[r_p] = \frac{2 p (1-p) }{N f(r_p)^2 }
\end{align}
for large $N$ where $f(r)$ is the pdf of $r_{xy}$. 
Assuming that $r_{xy}$ is Gaussian distributed and its variance is the one you measure, what is the variance in your estimate of $r_{0.95}$?
\end{frame}
\begin{frame}
7) Extra Credit: If you have time, do 1 through 3 for the Spearman and/or Kendall correlation coefficients.  There are Python functions for calculating them efficiently.  These are "rank statistics" that do not rely on any  assumption about the distribution of $X$ and $Y$ (in this case Gaussian) and are thus more widely applicable.  Do they indicate that the data in the home work is correlated?
\end{frame}

\end{document}